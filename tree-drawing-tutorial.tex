\documentclass[11pt]{article}

\usepackage{geometry}
\geometry{margin=1in}

\usepackage{parskip}

\usepackage[inline]{enumitem}

\usepackage{float}
\usepackage{subcaption}

\usepackage{graphicx}
\graphicspath{{figures/}}

\usepackage{fontspec}
\setmainfont[
  Ligatures={
    TeX,
    Common
  },
  BoldFont={AGaramond Pro Semibold}]{Adobe Garamond Pro}

\usepackage[missing={master}]{gitinfo2}

\renewcommand{\gitMark}{%
Branch: \gitBranch\,@\,\gitAbbrevHash{}
\textbullet{}
Release:\gitReln{}}

\usepackage{lastpage}

\usepackage{fancyhdr}
\pagestyle{fancy}

\usepackage{etoolbox}
\makeatletter
\patchcmd{\f@nch@head}{\rlap}{\color{gray}\small\sffamily\rlap}{}{}
\patchcmd{\f@nch@foot}{\rlap}{\color{gray}\small\sffamily\rlap}{}{}
\makeatother

\fancypagestyle{frontmatter}{%
  \renewcommand{\headrulewidth}{0pt}
  \renewcommand{\footrulewidth}{0pt}
  \lhead{}
  \chead{}
  \rhead{}
  \lfoot{}
  \cfoot{\gitMark}
  \rfoot{\thepage}}

\fancypagestyle{mainmatter}{%
  \renewcommand{\headrulewidth}{0pt}
  \renewcommand{\footrulewidth}{0pt}
  \lhead{}
  \chead{}
  \rhead{}
  \lfoot{}
  \cfoot{\gitMark}
  \rfoot{\thepage/\pageref*{LastPage}}}

\usepackage{minted}
\setminted{frame=single}

\usepackage[american]{babel}

\usepackage{csquotes}

\usepackage[
  backend=biber,
  bibstyle=biblatex-sp-unified,
  citestyle=sp-authoryear-comp,
  maxcitenames=2,
  maxbibnames=99]{biblatex}

\DeclareFieldFormat{doi}{%
  \textsc{doi}:
  \ifhyperref
  {\href{http://dx.doi.org/#1}{\nolinkurl{#1}}}
  {\nolinkurl{#1}}}

\addbibresource{tree-drawing-tutorial.bib}

\usepackage[linguistics]{forest}

\usepackage{xcolor}
\definecolor{darkred}{HTML}{B22613}

\usepackage{textglos}

\usepackage{gb4e}
\noautomath

\usepackage[colorlinks]{hyperref}
\hypersetup{
  linkcolor=darkred,
  citecolor=gray,
  urlcolor=cyan,
  bookmarksopen=true,
  bookmarksopenlevel=3,
  pdfborder={0 0 0}
}

\newcommand*{\Email}[1]{\href{mailto:#1}{\nolinkurl{#1}}}

\newcommand*{\Package}[1]{\texttt{#1}}
\newcommand*{\Software}[1]{\texttt{#1}}
\newcommand*{\Menu}[1]{\texttt{#1}}
\newcommand*{\File}[1]{\texttt{#1}}

\newcommand*{\Bracket}[1]{\textbf{\textcolor{red}{#1}}}

\newcommand*{\IE}{\textit{i.e.},}
\newcommand*{\ETC}{\textit{etc.}}

\title{\textsc{ling} 311 -- Tree drawing tutorial%
  \thanks{%
    A version of this tutorial was originally written by
    \href{https://www.diogoalmeida.net/}{Diogo Almeida} in fall 2006.
    The section on Microsoft Word greatly benefited from the input of
    the instructor of the course, Dr.\ Tonia Bleam.  This document was
    substantially revised by Adam Liter, the \textsc{ling} 311 TA for
    spring 2020.  Any questions or comments about this tutorial can be
    directed to either the current TA (via email or during office hours)
    or to Adam Liter (via email).  Also, the \File{.tex} that produced
    this \File{.pdf} is available at
    \href{https://github.com/adamliter/tree-drawing-tutorial}{\nolinkurl{https://github.com/adamliter/tree-drawing-tutorial}}.
    Pull requests, issues, and suggested improvements are welcome.%
  }}

\author{%
  \begin{tabular}{cc}
    \href{https://adamliter.org}{Adam Liter}\\
    \Email{liter@umd.edu}
  \end{tabular}}

\date{Last updated: \today}

\begin{document}

\maketitle

\thispagestyle{frontmatter}
\pagestyle{frontmatter}

\section{Introduction}
\label{sec:introduction}

This is a short tutorial on how to draw syntax trees on the computer
using commonly available resources.  We will be reviewing four easy ways
of producing trees:
%
\begin{enumerate}[label={(\roman*)}]
\item by using \href{https://www.overleaf.com}{\Software{Overleaf}}, a
  cloud-based \LaTeX\ system, and the package \Package{forest}
  (\S\ref{sssec:overleaf-and-forest});
\item by using a freely available JavaScript tool called
  \href{http://ironcreek.net/syntaxtree/}{\Software{jsSyntaxTree}}
  (\S\ref{sssec:jssyntaxtree});
\item by using the simple drawing functions of word processors like
  \Software{Microsoft Word}
  (\href{https://www.libreoffice.org/}{\Software{LibreOffice}} provides
  a free, open-source, cross-platform alternative to \Software{Microsoft
    Word}) (\S\ref{sssec:word-processors}); and
\item by using a free, open-source, cross-platform tool called
  \href{http://www.mapsofspeech.com/treeform/}{\Software{TreeForm}} that
  uses a What You See Is What You Get (WYSIWYG) interface
  (\S\ref{sssec:treeform}).
\end{enumerate}
%
Using the first option,
\href{https://www.overleaf.com}{\Software{Overleaf}}, is \emph{strongly
  encouraged}.  However, you are of course free to choose whatever
method you prefer, or even come up with your own, if you feel so
inclined.

\section{Tree drawing}
\label{sec:tree-drawing}

Both of the first two options (cf.~\S\ref{sssec:overleaf-and-forest} and
\S\ref{sssec:jssyntaxtree}) require using bracket notation for trees.
There are several resources for familiarizing yourself with bracket
notation.  First, there is \S3.1 of \href{https://msu.edu/~amunn/}{Alan
  Munn}'s
\href{https://msu.edu/~amunn/latex/overleaf-trees-forest.pdf}{\Software{Overleaf}
  tree drawing tutorial}.  Alan's tutorial shows trees that look more
like the trees we will be drawing by the end of the course as we develop
our syntactic theory.  I've adapted Alan's example below in
\S\ref{ssec:bracket-notation} for a tree that you might be more familiar
with.  Second, you can refer to \S2.3 of chapter~3 of the textbook for
this course, \textcite{carnie2002,carnie2007,carnie2013}; this starts on
p.~84 of the second edition and p.~95 of the third edition.  Third, you
can refer to
\href{https://umd.instructure.com/courses/1278736/files/55284562}{the
  video tutorial} put together by Adam Liter for using
\Software{Overleaf} and \Package{forest}.

\subsection{Bracket notation}
\label{ssec:bracket-notation}

Consider the tree in (\ref{ex:tree}).
%
\begin{exe}
  \ex{%
    \begin{forest}
      [S
        [NP
          [Det\\The]
          [Adj\\happy]
          [N\\giraffe]
        ]
        [VP
          [V\\saw]
          [NP
            [Det\\the]
            [N\\lion]
          ]
        ]
      ]
    \end{forest}}
    \label{ex:tree}
\end{exe}
%
To convert this tree in to bracket notation, we want include a pair of
opening and closing brackets (`\texttt{[}' and `\texttt{]}') for each
node in the tree.  Moreover, each pair of brackets gets a label, and the
label goes with the opening bracket (`\texttt{[}').  It works best to
draw your tree by hand on a piece of paper; then, when you go to convert
it into bracket notation, you should start drawing your tree from the
top, giving each node matching opening and closing brackets and a label.
In (\ref{ex:tree-bracketed}), I've shown the brackets in red.  Note that
there are four closing brackets at the end; this is because all nodes
require opening and closing brackets, so these four brackets correspond
to closing off the N (\xv{lion}) node, the NP node, the VP node, and the
S node.  Similarly, there are two closing brackets at the end of the
subject NP, which correspond to closing off the N (\xv{giraffe}) node
and the subject NP node.
%
\begin{exe}
  \ex{%
    \begin{forest}
      [\Bracket{[$_{\text{S}}$}S
        [\Bracket{[$_{\text{NP}}$}NP
          [\Bracket{[$_{\text{Det}}$}Det\\The\Bracket{]}]
          [\Bracket{[$_{\text{Adj}}$}Adj\\happy\Bracket{]}]
          [\Bracket{[$_{\text{N}}$}N\\giraffe\Bracket{]]}]
        ]
        [\Bracket{[$_{\text{VP}}$}VP
          [\Bracket{[$_{\text{V}}$}V\\saw\Bracket{]}]
          [\Bracket{[$_{\text{NP}}$}NP
            [\Bracket{[$_{\text{Det}}$}Det\\the\Bracket{]}]
            [\Bracket{[$_{\text{N}}$}N\\lion\Bracket{]]]]}]
          ]
        ]
      ]
    \end{forest}}
    \label{ex:tree-bracketed}
\end{exe}
%
I encourage you to refer to
\href{https://umd.instructure.com/courses/1278736/files/55284562}{the
  video tutorial} put together by Adam Liter for using
\Software{Overleaf} and \Package{forest}; this tutorial also covers the
usage of bracket notation, and it may be easier to follow a video
demonstration rather than reading the explanation given here.

Finally, note that this bracket notation can be used with both
\Software{Overleaf} and \Software{jsSyntaxTree}; however, there are some
slight differences about the bracket notation that only work in one case
and not the other.  This only applies to more advanced tree drawing,
though.  Almost everything in this tutorial should work just fine with
both \Software{Overleaf} and \Software{jsSyntaxTree} (the one exception
deals with how line breaks are handled; see \S\ref{sssec:jssyntaxtree}).

\subsection{Tree drawing programs}
\label{ssec:tree-drawing-programs}

\subsubsection{\Software{Overleaf} and \Package{forest}}
\label{sssec:overleaf-and-forest}

\href{https://www.overleaf.com}{\Software{Overleaf}} is a cloud-based
\LaTeX\ system; \LaTeX\ is a document preparation system.  Using \LaTeX\
may look a bit like programming, but you should not be intimidated.  We
are only going to be writing a very basic \LaTeX\ document, and you do
not need to install \LaTeX\ on your own computer since we are using
\Software{Overleaf}.  You only need to create an account on
\Software{Overleaf}.%
\footnote{%
  If you are interested in learning more about \LaTeX, one good resource
  is this \href{https://bit.ly/latex-workshop}{\LaTeX\ tutorial}.}

You are encouraged to just watch the
\href{https://umd.instructure.com/courses/1278736/files/55284562}{the
  video tutorial} put together by Adam Liter for using
\Software{Overleaf} and \Package{forest}; however, in what follows, I
summarize the video tutorial.

For each tree that you want to draw, you'll need to create a new project
on \Software{Overleaf}.  You should start each new project with the
setup given in Listing~\ref{lst:standalone-forest}.  You can just copy
and paste this in to your new project; alternatively, there is an
Overleaf template called ``Linguistics: Syntax tree drawing template
using forest'', which you can search for and use as the basis of a new
project.

\begin{listing}[H]
  \centering
  \begin{minted}{latex}
\documentclass[varwidth=true, border=5pt]{standalone}

\usepackage[linguistics]{forest}

\begin{document}
\begin{forest}
[S
]
\end{forest}
\end{document}
  \end{minted}
  \caption{A template for drawing trees with \Software{Overleaf} and \Software{forest} using the \Package{standalone} document class}
  \label{lst:standalone-forest}
\end{listing}

Everything between the \mintinline{latex}|\documentclass{...}|
declaration and \mintinline{latex}|\begin{document}| is called the
preamble of the document; this is where you can load packages and
define commands.  To draw a tree, you only need to load the
\Package{forest} package and specify the optional
\Package{linguistics} argument.

The rules for using the \Package{forest} package are straightforward.
%
\begin{enumerate}[label={(\roman*)}]
\item Every tree goes in between a line with
  \mintinline{latex}|\begin{forest}| and a line with
  \mintinline{latex}|\end{forest}|.
\item Every opening bracket needs a matching closing bracket.
\item In general, whitespace (spaces and line breaks) does not matter
  inside the \Package{forest} environment, so you can break your
  bracketed tree across multiple lines; however, \emph{there are two
    important exceptions}:
\begin{enumerate*}[label={(\alph*)}]
\item there can be no blank lines in the \Package{forest} environment;
  and
\item the label for a node must appear immediately next to the opening
  bracket (\IE{} `\mintinline{latex}|[S|' is acceptable but
  `\mintinline{latex}|[ S|' is not).
\end{enumerate*}
\end{enumerate}
%
A final document that produces the tree in (\ref{ex:tree}) is given in
Listing~\ref{lst:full-forest-example}.

\begin{listing}[H]
  \centering
  \begin{minted}{latex}
\documentclass[varwidth=true, border=5pt]{standalone}

\usepackage[linguistics]{forest}

\begin{document}
\begin{forest}
[S
  [NP
    [Det\\The]
    [Adj\\happy]
    [N\\giraffe]
  ]
  [VP
    [V\\saw]
    [NP
      [Det\\the]
      [N\\lion]
    ]
  ]
]
\end{forest}
\end{document}
 
  \end{minted}
  \caption{A complete document that produces the tree in (\ref{ex:tree}) using \Software{Overleaf} and \Package{forest}}
  \label{lst:full-forest-example}
\end{listing}

Once you've finished drawing the tree, you can click the ``Download
PDF'' button in \Software{Overleaf} to download the resulting tree and
insert it into a document, such as a \Software{Microsoft Word} document,
if that is how you are writing up your homework.

\subsubsection{\Software{jsSyntaxTree}}
\label{sssec:jssyntaxtree}

Another tree drawing option that uses bracket notation is called
\href{http://ironcreek.net/syntaxtree/}{\Software{jsSyntaxTree}}.

If you go to the website, all you need to do is input the tree in
bracket notation.  You can then right click on the resulting image, save
it to your computer, and insert it into a document, such as a
\Software{Microsoft Word} document, if that is how you are writing up
your homework.

In general, you can use the same bracket notation for
\Software{jsSyntaxTree} as you would with \Software{Overleaf} and
\Package{forest}.  One exception is that you cannot use two backslashes
(\mintinline{latex}|\\|) for a line break with \Software{jsSyntaxTree};
instead, you just enter the word that is at the leaf of the tree after
the label and one space.  You can draw the tree in (\ref{ex:tree}) using
\Software{jsSyntaxTree} by pasting the bracket notation in
Listing~\ref{lst:jssyntaxtree-bracketing} into the website.

\begin{listing}[H]
  \centering
  \begin{Verbatim}[frame=single]
[S
  [NP
    [Det The]
    [Adj happy]
    [N giraffe]
  ]
  [VP
    [V saw]
    [NP
      [Det the]
      [N lion]
    ]
  ]
]
  \end{Verbatim}
  \caption{Bracket notation for producing the tree in (\ref{ex:tree}) using \Software{jsSyntaxTree}}
  \label{lst:jssyntaxtree-bracketing}
\end{listing}

\subsubsection{Word processors}
\label{sssec:word-processors}

Drawing trees in a word processor involves two simple stages:
\begin{enumerate*}[label={(\alph*)}]
\item typing the structure of the tree; and
\item connecting the nodes you typed using the line drawing function.
\end{enumerate*}
We are going to go through these below.

The first thing you need to do is type the structure of the tree
(cf.~Figure~\ref{fig:word-processor-1}).  You can use tabs to space the
nodes out, and you will probably need to skip at least one line every
time you finish with a structure level. (Notice in
Figure~\ref{fig:word-processor-1} that there are actually three blank
lines between the subject NP node and the Det, Adj, and N nodes inside
that NP; this is because of needing to leave space for the VP
structure.)
%
\begin{figure}[H]
  \centering
  \includegraphics[width=0.6\textwidth]{word-processor-1}
  \caption{Typing the structure of the tree}
  \label{fig:word-processor-1}
\end{figure}
%
Next, you will need to start inserting lines.  You can do this by going
to the \Menu{Insert} tab (or menu item).  From there, you can select the
\Menu{Shapes} dropdown menu and choose the simple \Menu{Line} shape
(cf.~Figure~\ref{fig:word-processor-2}).
%
\begin{figure}[H]
  \centering
  \includegraphics[width=0.6\textwidth]{word-processor-2}
  \caption{Inserting a line}
  \label{fig:word-processor-2}
\end{figure}
%
Now you are ready to connect the nodes.  Just draw lines between them
(cf.~Figure~\ref{fig:word-processor-3}).  Sometimes you will need to
make minor adjustments either on the position of the nodes or on the
position of the lines so the tree will look ok.
%
\begin{figure}[H]
  \centering
  \includegraphics[width=0.6\textwidth]{word-processor-3}
  \caption{Drawing the lines}
  \label{fig:word-processor-3}
\end{figure}
%
The final result should look something like in
Figure~\ref{fig:word-processor-4}.
%
\begin{figure}[H]
  \centering
  \includegraphics[width=0.6\textwidth]{word-processor-4}
  \caption{Final tree from using a word processor}
  \label{fig:word-processor-4}
\end{figure}

The biggest downside to using a word processor to draw your trees is
that you'll need to make sure that the tree stays together as you add
writing above and below the tree; you'll also need to make sure that the
tree does not get separated across multiple pages of the document.  The
best way to do this is to select all but the last line of your tree,
right click, select \Menu{Paragraph}, then click on the \Menu{Line and
  Page Break} tab, and select the \Menu{Keep with next} option
(cf.~Figure~\ref{fig:word-processor-5-and-6}).  This should keep your
tree together.
%
\begin{figure}[H]
  \centering
  \begin{subfigure}[b]{0.45\textwidth}
    \centering
    \includegraphics[width=0.9\textwidth]{word-processor-5}
    \caption{Highlight and select \Menu{Paragraph}}
    \label{subfig:word-processor-5}
  \end{subfigure}
  \begin{subfigure}[b]{0.45\textwidth}
    \centering
    \includegraphics[width=0.9\textwidth]{word-processor-6}
    \caption{Select \Menu{Keep with next}}
    \label{subfig:word-processor-6}
  \end{subfigure}
  \caption{Keeping a tree together when drawn with a word processor}
  \label{fig:word-processor-5-and-6}
\end{figure}

\subsubsection{\Software{TreeForm}}
\label{sssec:treeform}

Finally, the last option is to use
\href{http://www.mapsofspeech.com/treeform/}{\Software{TreeForm}}.
\Software{TreeForm} uses a WYSIWYG approach to drawing trees.  As such
it is fairly intuitive, and we will not cover it in detail here.  There
is a video tutorial at the linked website that you can watch.

In brief, you will need to download the program, install it, draw your
tree using the interface, and then export it as an image which you can
insert into a document, such as a \Software{Microsoft Word} document, if
that is how you are writing up your homework.

Note that \Software{TreeForm} is somewhat of an outdated Java
application.  As of July 2019, the author of the software has begun to
try to update and modernize it.  However, because of it being somewhat
outdated, you are discouraged from using this option.  It is unlikely
that the TA will be able to help you all that much with this option if
you get stuck.

\printbibliography

\end{document}

% Local Variables:
% coding: utf-8
% mode: latex
% TeX-engine: luatex
% TeX-master: t
% LaTeX-biblatex-use-Biber: t
% TeX-command-extra-options: "-shell-escape"
% End:
